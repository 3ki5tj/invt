\documentclass[reprint]{revtex4-1}
\usepackage{amsmath}
\begin{document}


\title{Optimal updating factor in Wang-Landau and metadynamics simulations}

\maketitle


\section{Background}



\subsection{Flat-distribution sampling}



Given a system,
consider the problem of computing
the distribution, $p^*_i$
along a discrete quantity $i$.
%
The discrete nature of $i$ is natural
in discrete models.
%
For example, $i$ can be the energy $E$
in lattice models; or the temperature index
in a simulated tempering simulation.
%
For a continuous quantity, $z$,
which is typical in molecular systems,
we can also discretize $z$
such that the integer $i$ represents
the index of a small interval, or a bin, $(z, z + dz)$.
%
In both cases,
the distribution can be normalized as
$\sum_{i = 1}^n p^*_i = 1$.



For a large system,
the distribution $p^*_i$ is often
localized around some $i_0$,
%
and to find out the global property,
it is often desirable to carry out
a biased sampling that targets
a wider distribution $p_i$.
%
%Here, we refer to simulations that target
%a flat or nearly flat distribution
%as entropic or multicanonical sampling.



To do so, we need to introduce a bias potential $\phi_i$,
such that the target distribution is changed to
%
\begin{equation}
  \pi_i \propto p^*_i \, e^{-\phi_i}.
  \label{eq:pi_p_phi1}
\end{equation}
%
Upon normalization $\sum_{i = 1}^n \pi_i = 1$,
we get
%
\begin{equation}
  \pi_i =
  \frac{ p^*_i \, e^{-\phi_i} }
  { \sum_{j = 1}^n p^*_j \, e^{-\phi_j} }.
  \label{eq:pi_p_phi}
\end{equation}
%
Particularly,
to achieve a flat distribution $\pi_i$,
the bias potential $\phi_i$
must coincide with $\log p^*_i$,
up to an additive constant.



Generally, we may wish to target
a slowly-varying distribution, $p_i$,
by adjusting the bias potential $\phi_i$
on the fly.
%
For example,
we can monitor the histogram accumulated over
a period of time and use it for $\pi_i$ in
Eq. \eqref{eq:pi_p_phi1},
and then update
$\phi_i$ according to
$$
\phi_i^{\mathrm{new}}
=
\phi_i^{\mathrm{old}}
-
\log \frac{ p_i } { \pi_i }.
$$
%
%From Eq. \eqref{eq:pi_p_phi1},
%it is clear that to achieve the target distribution
%$\pi_i = p_i$ requires
%%
%\begin{equation}
%\log p^*_i
%=\phi_i + \ln p_i + c
%\label{eq:bias_pmf}
%\end{equation}
%%
%%
%In the end of the simulation,
%the PMF, $\phi^*_i$,
%can be deduced from Eq. \eqref{eq:bias_pmf}.
%
The above inversion process can be inconvenient
in practice,
for we need to make the period
as long as a multiple of
the autocorrelation time,
which is difficult to estimate for an unknown system.



\subsection{Wang-Landau algorithm}



The Wang-Landau (WL) algorithm is a convenient technique
of obtaining the desired bias potential $\phi_i$
on the fly.
%
To simplify the notation, we introduce
a shifted bias potential
%
\begin{equation}
  v_i \equiv \phi_i - \ln p^*_i + \ln p_i.
  \label{eq:v_def}
\end{equation}
%
%
and in $v_i$, Eq. \eqref{eq:pi_p_phi}
becomes
%
\begin{equation}
  \pi_i = \frac{ p_i e^{-v_i} }
  { \sum_{j = 1}^n p_j e^{-v_j} }
  \propto p_i \, e^{-v_i}.
  \label{eq:pi_p_v}
\end{equation}
%
According to Eq. \eqref{eq:pi_p_phi1},
$v_i$ should approach a constant of $i$
upon convergence (i.e., when $\pi_i \approx p_i$).



In the WL algorithm, $v_i$ are updated
in each MC step $t$,
and $v_i(t)$ as a function of $t$
is updated as
%
\begin{equation}
  v_i(t+1)
  =
  v_i(t)
  +
  \delta_{i, \, i(t)}
  \frac{ \alpha(t) } { p_i }.
  \label{eq:wl_update}
\end{equation}
%
where $i(t)$ is the bin at step $t$,
and $\alpha(t)$ is the updating magnitude.
%
Note that the updating scheme only applies
to the $v_i$ at the current bin.
%The values of $v_j(t)$ at $j \ne i(t)$
%are kept unchanged.
%
One can show that with a constant $\alpha(t) = \alpha > 0$,
the distribution collected from
the trajectory is identical to $p^*_i$.
%
However, the updating scheme
makes $v_i(t)$ a fluctuating quantity,
and Eq. \eqref{eq:pi_p_v} no longer holds
at all times.
%%
%The source of the deviation is two-fold.
%%
%First, since $v_i(t)$ is updated continuously,
%there is a random noise that is proportional
%to $\sqrt \alpha$.
%%
%Second, there is a systematic error
%that comes from the updating dynamics itself,
%as it breaks the Markovian nature
%that underlies Eq. \eqref{eq:pi_p_v}.
%%
%
Thus, one has to decrease $\alpha$ over time
to reduce the error of $v_i$,
in order to get the unbiased distribution
$\ln p_i^*$ via Eq. \eqref{eq:v_def}.





In the original WL scheme,
the updating magnitude $\alpha$ (or $\ln f$
in the original paper) is kept as a constant
for a period of time,
which is referred to as a stage below.
%
The histogram collected in the stage is monitored.
%
Once the histogram is sufficiently flat,
we are allowed to enter a new stage
with a reduced $\alpha$
(usually half as large as
that in the previous stage).
%
This scheme works well for early stages.
%
However, in later stages, it tends to reduce $\alpha$
too quickly, making the asymptotic error
saturate.



\subsection{$1/t$ formula}



A more effective way
of updating the $\alpha(t)$
is to discard the stage-wise setup,
and to follow the formula
%
\begin{equation}
  \alpha(t) = \frac{1}{t},
  \label{eq:alpha_invt}
\end{equation}
%
where $t$ is the number of steps
from the beginning of the MC simulation,
which shall be referred to as the ``time'' below.
%
This surprisingly simple formula has attracted
several studies, notably the one by Zhou.
%
There are, however, some reservations about
the constant of proportionality:
the optimal formula for $\alpha(t)$
might be $C/t$ with a constant $C$
different from $1.0$.



Here we shall derive the optimal $\alpha(t)$
for the WL algorithm, Eq. \eqref{eq:wl_update}.
%
Further, we shall generalize the result
to a class of multiple-bin updating schemes,
in which an update affects
not only the current bin $i(t)$,
but also a few neighboring ones.
%(such schemes are designed
%to maintain a smooth bias potential).
%
We shall show that
the constant of proportionality, $1$,
in Eq. \eqref{eq:alpha_invt}
is optimal for the single-bin updating scheme,
Eq. \eqref{eq:wl_update}
(as in the WL algorithm),
but not necessarily so
for a multiple-bin updating scheme.
%
Below, we discuss the single-bin updating scheme
in Section \ref{sec:single-bin},
and the multiple-bin case
in Section \ref{sec:multiple-bin}.



\section{\label{sec:single-bin}
Single-bin updating scheme}



In this section,
we shall derive the optimal $\alpha(t)$
for the single-bin updating scheme,
Eq. \eqref{eq:wl_update}.
%
To do so,
we shall first express the error of $v(t)$
as a functional of $\alpha(t)$,
and then find the optimal value
by functional variation.



\subsection{Differential equation}



We first approximate Eq. \eqref{eq:wl_update}
by a differential equation
%
\begin{equation}
  \dot v_i(t)
  =
  h_i(t) \frac{ \alpha(t) } { p_i },
  \label{eq:vt_diffeq}
\end{equation}
%
where
$\dot v_i(t) \equiv dv_i(t)/dt$,
%
and $h_i(t) = \delta_{i, i(t)}$
is the instantaneous histogram,
which is equal to $1.0$
for the current bin $i(t)$
or zero otherwise.



Next, we split $h_i(t)$ into a deterministic part
and a noise
%
\begin{equation}
  h_i(t) = \langle h_i(t) \rangle + \zeta_i(t).
  \label{eq:h_split}
\end{equation}
%
Here, the deterministic part can be related
to the ``ensemble average'' of $h_i$.
%
The ensemble consists of many similar simulation copies
that have experienced the same schedule $\alpha(t)$
and have reached the same $v_i(t)$
at time $t$.
%
The initial states and random numbers of the copies
may, however, be different.



For sufficiently small $\alpha(t)$,
the sampling process is approximately
Markovian of a finite order,
and we may assume Eq. \eqref{eq:pi_p_v}
for the deterministic part
%
\begin{equation}
  \langle h_i(t) \rangle
  \approx
  \frac{ p_i \, e^{-v_i} }
  { \sum_{j = 1}^n p_j \, e^{-v_j} }.
  \label{eq:h_ave}
\end{equation}



For the noise part, we have
$\langle \zeta_i(t) \rangle = 0$.
%
%The noise is not necessarily white.
Unless specified otherwise, we assume
a general noise with the correlation
function depending on the time difference:
%
\begin{equation}
  \langle \zeta_i(t) \, \zeta_j(t') \rangle
  =
  \sigma_{ij}(t - t'),
  \label{eq:zeta_zeta_correlation}
\end{equation}
%
where $\sigma_{ij}(t)$ is an even function of $t$
that vanishes at large $t$.



\subsection{Linear approximation}



To proceed, we first observe that
the average $\bar v = \sum_{i = 1}^n p_i \, v_i$
increases steadily over time:
%
\begin{equation}
\frac{ d \bar v } { d t }
=
\sum_{i = 1}^n p_i \dot v_i
=
\alpha(t) \sum_{i = 1}^n h_i(t) = \alpha(t).
\label{eq:dvbardt}
\end{equation}
%
However, the difference between $v_i$ and $\bar v$,
%
\begin{equation}
  x_i \equiv v_i - \bar v = v_i - \sum_{j = 1}^n p_j \, v_j,
  \label{eq:x_def}
\end{equation}
%
is expected to be small in the asymptotic regime,
and we can expand Eq. \eqref{eq:h_ave} as
\begin{align}
\langle h_i(t) \rangle
&\approx
\frac{ p_i \, e^{- x_i} }
{ \sum_{ j = 1}^n p_j \, e^{- x_j} }
\approx
\frac{ p_i ( 1 - x_i ) }
{ \sum_{ j = 1}^n p_j (1 - x_j) }
\notag \\
&\approx
p_i \, \left(
  1 - x_i + \sum_{j=1}^n p_j \, x_j
\right)
=
p_i \, (1 - x_i),
\label{eq:hdet_linearize}
\end{align}
where we have used $\sum_{j=1}^n p_j = 1$,
and
%
\begin{equation}
  \sum_{i = 1}^n p_i \, x_i = 0,
  \label{eq:px_sum}
\end{equation}
which follows directly from Eq. \eqref{eq:x_def}.
%
With Eqs.
\eqref{eq:vt_diffeq},
\eqref{eq:h_split},
\eqref{eq:dvbardt},
and
\eqref{eq:hdet_linearize},
we get a set of decoupled equations
%
\begin{equation}
  \dot x_i(t)
  =
  -\alpha(t) \, \left[ x_i(t) - \frac{ \zeta_i(t) } { p_i } \right].
  \label{eq:dxdt_WL}
\end{equation}



The total error of the bias potentials can be written in
terms of $x_i$,
\begin{equation}
\mathcal E
=
\sum_{i = 1}^n p_i \, \langle x_i^2 \rangle
=
\sum_{i = 1}^n p_i \, \left\langle (v_i - \bar v)^2 \right\rangle,
\label{eq:error_sum}
\end{equation}
where
$\bar v = \sum_{i = 1}^n p_i v_i$.
%
Our aim is to find the $\alpha(t)$
that minimizes Eq. \eqref{eq:error_sum}.
%
To do so, we shall first consider
a simpler one-variable problem.



\subsection{One-variable problem}



Consider the following equation
of a single variable:
%
\begin{equation}
\dot x(t) = -\alpha(t) \, \lambda \left[ x(t) - \xi(t) \right],
\label{eq:dxdt_alpha}
\end{equation}
%
where $\xi(t)$ is a generalized noise
that is invariant under time translation:
%
\begin{equation}
\left\langle \xi(t) \, \xi(t') \right\rangle
=
\kappa(t - t').
\label{eq:noise_correlation}
\end{equation}
%
Additionally, we require
\begin{equation}
  \lim_{t \rightarrow \pm\infty} \kappa(t) = 0.
  \label{eq:kappat_limit}
\end{equation}
%
For example, for a white noise,
$\kappa(t)$ is proportional to
Dirac's $\delta$-function, $\delta(t)$.
%
We wish to show that the optimal $\alpha(t)$
of minimizing $\langle x^2(t) \rangle$ at long times
is given by
%
\begin{equation}
  \alpha(t) = \frac{1}{\lambda \, (t + t_0)}.
\label{eq:alpha_opt}
\end{equation}



To do so, we first recall
the formal solution of Eq. \eqref{eq:dxdt_alpha}:
%
\begin{equation}
x(t) = x(0) \, e^{-\lambda \, q(t)}
+ \int_0^t \dot u(t') \, \xi(t') \, dt',
\label{eq:xt_solution}
\end{equation}
%
where,
%
\begin{equation}
q(t) \equiv \int_0^t \alpha(t') \, dt',
\label{eq:qt_definition}
\end{equation}
%
and
%
\begin{align}
u(q(t'))
&\equiv
e^{-\lambda \, q(t) + \lambda \, q(t')}.
\label{eq:ut_definition}
\end{align}


We shall further demand that
%
\begin{equation}
  \lim_{t \to \infty} q(t) \to \infty.
  \label{eq:qt_limit}
\end{equation}
%
Then, for a large time $t$,
the first term on the right-hand side
of Eq. \eqref{eq:xt_solution} can be neglected, and
%
\begin{align}
\left\langle x^2(t) \right\rangle
%&=
%\int_0^t \int_0^t \dot u(t') \, \dot u(t'')
%    \left\langle \xi(t') \xi(t'') \right\rangle dt'' \, dt'
%\notag
%\\
&=
\int_0^t \int_0^t
  \dot u(q(t')) \, \dot u(q(t'')) \,
  \kappa(t' - t'') \, dt'' \, dt'.
\label{eq:x2t_average}
\end{align}



Variating this expression and
using the Euler-Lagrange equation yields
$$
\begin{aligned}
0
&=
\frac{d}{d\tau} \int_0^t \dot u(q(t')) \, \kappa(t' - \tau) \, dt'
= \int_0^t \ddot u(q(t')) \, \kappa(t' - \tau) \, dt',
\end{aligned}
$$
where we have dropped the boundary terms
by using Eq. \eqref{eq:kappat_limit}.
%
%In fact, we can repeat the process $n$ times, and
%$$
%\int_0^t u^{(n)}(t') \, \kappa(t' - \tau) \, dt' = 0,
%\qquad (n \ge 1)
%$$
%
In order for this to hold for any $\tau$,
we must have
%
\begin{equation}
\ddot u(q(t')) = 0,
\qquad
\mathrm{or}
\;\;
\dot u(q(t')) = c,
\label{eq:ddu_eq_0}
\end{equation}
%
where $c$ is a constant of $t'$.
%
Using Eq. \eqref{eq:ut_definition}
we get
$$
e^{-\lambda \, q(t) + \lambda \, q(t')}
=
c \, (t' + t_0),
$$
where $t_0$ is a constant.
%
Taking the logarithm, and differentiating this with respect to $t'$
yields Eq. \eqref{eq:alpha_opt}.



\subsection{Optimal $\alpha(t)$ for the single-bin updating scheme}



Going back to the problem of
minimizing Eq. \eqref{eq:error_sum}
for the single-bin updating scheme,
we find from Eq. \eqref{eq:dxdt_WL}
that with $\lambda = 1$,
Eq. \eqref{eq:alpha_opt}
will optimize each term $\langle x_i^2 \rangle$,
and hence the sum.
%
Thus, Eq. \eqref{eq:alpha_invt} is optimal
up to a constant $t_0$ defining the origin of $t$.



\section{\label{sec:multiple-bin}
Multiple-bin update}



We now extend the above analysis to
the multiple-bin updating scheme.
%
In this case,
upon a visit to bin $j$,
we update not only $v_j$,
but also $v_i$ for a few neighboring bins $i$.
%
The update magnitudes are given by a matrix $\mathbf w$
with elements $w_{ij}$.
%
Then, Eq. \eqref{eq:vt_diffeq} is generalized to
\begin{equation}
  \dot v_i(t) =
  \sum_{j=1}^n \alpha(t) \, h_j(t) \frac{ w_{ij} } { p_j }.
  \label{eq:vt_diffeq_mbin}
\end{equation}


The procedure of optimization is largely
the same as the single-bin case,
whereas the matrix nature brings about
some difficulties that require
certain restrictions and approximations.



\subsection{Updating matrix $\mathbf w$}



The matrix $\mathbf w$ is not arbitrary.
%
A necessary condition to sample desired distribution
$\mathbf p = (p_1, \dots, p_n)$
is that in Eq. \eqref{eq:vt_diffeq_mbin}
when $h_j(t)$ coincides with $p_j$,
the rate of change $\dot v_i(t)$
should be independent of $i$.
%
This allows $v_i(t)$ to approach a constant
asymptotically.
%
Thus, by a proper scaling of $\alpha(t)$,
we may write this condition as
%
\begin{equation}
  \sum_{j = 1}^n w_{ij} = 1.
  \label{eq:w_sumj}
\end{equation}
%
This means $(1, \dots, 1)^T$
is a right eigenvector of $\mathbf w$
with eigenvalue $1$.
%
Thus, the transpose $\mathbf w^T$
resembles a transition matrix,
although the matrix elements can be negative.



To simplify the following discussion,
we shall further impose the
detailed balance condition:
%
\begin{equation}
  p_i \, w_{ij} = p_j \, w_{ji}.
  \label{eq:w_detailedbalance}
\end{equation}
%
It follows that
\begin{equation}
  \sum_{i = 1}^n p_i \, w_{ij}
  =
  \sum_{i = 1}^n p_j \, w_{ji}
  = p_j,
  \label{eq:w_balance}
\end{equation}
%
i.e., $\mathbf p$ is a left eigenvector of
$\mathbf w$ with eigenvalue $1$.



Further, by Eq. \eqref{eq:w_detailedbalance},
we can define a symmetric matrix $\hat{\mathbf w}$
as $\hat w_{ij} = \sqrt{p_i/p_j} \, w_{ij}$
that can be diagonalized
with a set of orthonormal eigenvectors:
%
$\sum_{i = 1}^n U_{ki} \, \hat w_{ij} = \lambda_k \, U_{kj}$.
%
Thus,
in terms of diagonalizing $\mathbf w$, we have
\begin{equation}
  \sum_{i = 1}^n T_{ki} \, w_{ij} = \lambda_k \, T_{kj},
  \label{eq:T_w}
\end{equation}
where
$T_{ij} = \sqrt{p_j} \, U_{ij}$ satisfies
the orthonormal conditions:
%
\begin{align}
\sum_{k = 1}^n T_{ki} \, T_{kj}
&= \delta_{ij} \, p_i,
\label{eq:T_orthonormal_cols}
\\
\sum_{k = 1}^n \frac{ T_{ik} \, T_{jk} }{ p_k }
&= \delta_{ij}.
\label{eq:T_orthonormal_rows}
\end{align}



We emphasize that, in a stable updating scheme,
all eigenvalues $\lambda_k$ must be nonnegative,
for otherwise there is a fluctuating mode
that can increase indefinitely.



\subsection{Linearization and decoupling}



We can now simplify Eq. \eqref{eq:vt_diffeq_mbin}.
%
By using Eq. \eqref{eq:w_balance},
we can show that
Eqs. \eqref{eq:dvbardt} and \eqref{eq:px_sum}
remain true.
%
%Thus, for Eq. \eqref{eq:x_def}, we have
%
$$
\begin{aligned}
\dot x_i
&= \alpha(t) \sum_{j=1}^n w_{ij}
\left( \frac{ h_j } { p_j }  - 1 \right)
\\
&=
-\alpha(t) \sum_{j = 1}^n
w_{ij} \left[ x_j(t) - \frac{\zeta_j (t)}{p_j} \right],
\end{aligned}
$$
where
we have expanded the right-hand side
in linear terms of $x_j$ in the second step.


Now by diagonalizing the matrix $\mathbf w$
using matrix defined in Eqs. \eqref{eq:T_w},
\eqref{eq:T_orthonormal_cols},
and
\eqref{eq:T_orthonormal_rows},
we get a set of decoupled equations
%
\begin{equation}
\dot y_i(t)
=
-\alpha(t) \, \lambda_i
[y_i(t) - \eta_i(t)],
\label{eq:yt_diffeq}
\end{equation}
%
where
\begin{align}
  y_i &= \sum_{j=1}^n T_{ij} \, x_j,
  \label{eq:y_def}
  \\
  \eta_i &= \sum_{j=1}^n T_{ij} \frac{ \zeta_j}{ p_j}.
  \label{eq:eta_def}
\end{align}



\subsection{Minimization of the error function}



The error function can be obtained
from Eq. \eqref{eq:T_orthonormal_cols}
as
\begin{align}
  \mathcal E
  &\equiv
  \sum_{i = 1}^n p_i \, x_i^2
  =
  \sum_{i, j, k=1}^n T_{ki} \, T_{kj} \, x_i \, x_j
  =
  \sum_{k = 1}^n y_k^2.
  \label{eq:y2_sum}
\end{align}
%
If we arrange $y_k$ such that
the mode $y_1 = \sum_{j=1}^n p_j \, x_j$,
(corresponding to $\lambda_1 = 1$),
then $y_1$ vanishes identically
by Eq. \eqref{eq:px_sum},
and the sum of $y_k$ in Eq. \eqref{eq:y2_sum}
can start from $k = 2$.



In analogous to Eq. \eqref{eq:x2t_average},
we can rewrite Eq. \eqref{eq:y2_sum} as
%
\begin{align}
  \mathcal E
  %&= \sum_{i = 2}^n \langle y_i^2 \rangle
  %\notag
  %\\
  &=
  \int_0^t \int_0^t
  \sum_{i = 2}^n
  \dot u_i(q(t')) \,
  \dot u_i(q(t'')) \,
  \kappa_i(t' - t'') \, dt' \, dt'',
  \notag
  %\label{eq:y2_int}
\end{align}
%
where
\begin{align*}
  u_i(q(t'))
  &\equiv
  e^{\lambda_i \, [q(t') - \, q(t)]},
  \\
  \kappa_i(t - t')
  &\equiv
  \left\langle \eta_i(t) \, \eta_i(t') \right\rangle,
\end{align*}
%
and the extremal condition of $\mathcal E$ is given by
%
\begin{align}
\sum_{i=2}^n
u_i'(q(\tau))
\int_0^t
\ddot u_i(q(t')) \, \kappa_i(t' - \tau) \, dt' = 0.
\label{eq:optimal_mbin1}
\end{align}
%
If all $\lambda_i$, hence all $u_i$, are the same,
the above condition can be satisfied
by Eq. \eqref{eq:ddu_eq_0},
which is the single-bin case.
%
The general solution of Eq. \eqref{eq:optimal_mbin1}
is harder to obtain.
%



\subsection{Optimal $\alpha(t)$}


To proceed, we shall further assume perfect sampling,
%
\begin{equation}
  \langle \zeta_i(t) \, \zeta_j(t') \rangle
  =
  p_i \, \delta_{ij} \, \delta(t - t').
  \label{eq:zz_perfect}
\end{equation}
%
Thus, from Eqs. \eqref{eq:T_orthonormal_rows} and \eqref{eq:eta_def},
we get
%
\begin{equation}
  \kappa_i(t - t')
  =
  \sum_{j,k = 1}^n
  \frac{ T_{ij} } { p_j }
  \frac{ T_{ik} } { p_k }
  \langle \zeta_j(t) \, \zeta_k(t') \rangle
  =
  \delta(t - t').
  \label{eq:kappa_perfect}
\end{equation}
%
Thus, Eq. \eqref{eq:optimal_mbin1}
is simplified as
%
\begin{align}
  \sum_{i=1}^n
  \ddot u_i(q(\tau)) \, u'_i(q(\tau)) = 0.
  \label{eq:optimal_mbin2}
\end{align}
%
%We can study two limiting cases.
%
For $\tau \ll t$,
the dominant contribution comes from
the term(s) with the smallest eigenvalue $\lambda_i = \lambda_n$,
and
$$
\ddot u_n(\tau) = 0,
\quad
\dot u_n(\tau) = c,
\quad
\mathrm{and}\;
u_n(\tau) = c \, (\tau + t_0),
$$
which means
$$
\alpha(\tau) \approx \frac{1}{\lambda_n \, (\tau + t_0)}.
$$



\subsection{Comparison of updating schemes}



A more direct, and often more accurate, approach
is to assume the functional form
\begin{equation}
\alpha(\tau) = \frac{1}{\lambda \, (\tau + t_0) },
\label{eq:alpha_trial}
\end{equation}
and find the optimal choice of $\lambda$.
%
Using Eqs. \eqref{eq:kappa_perfect}, \eqref{eq:alpha_trial}
in Eq. \eqref{eq:y2_sum}, we get
%
\begin{equation}
\mathcal E
=
\int_0^t
\sum_{i=0}^n \dot u_i^2(q(t')) \, dt'
\stackrel{t \to \infty}
{=\joinrel=\joinrel=}
\frac{1}{t}
\sum_{i = 2}^n
\frac{ \lambda_i^2 } { \lambda \, (2 \, \lambda_i - \lambda) },
\label{eq:y2_sum_perfect_invt}
\end{equation}
%
and the optimal $\lambda$
can be found from the smallest root of
%
\begin{equation}
\sum_{i = 2}^n
\frac{ \lambda_i - \lambda }
{ \left(2 - \lambda/ \lambda_i \right)^2 }
= 0.
\end{equation}



Under the approximation, we have
$$
\frac{ \lambda_i^2 }{ \lambda \, (2 \lambda_i - \lambda) } \ge 1,
$$
with the equality achieved only at $\lambda = \lambda_i$,
we see the from Eq. \eqref{eq:y2_sum_perfect_invt}
that the minimal error is achieved with
\begin{equation}
  \lambda_2 = \dots = \lambda_n = \lambda.
\end{equation}
%
and the updating matrix $\mathbf w$ satisfying this condition
is essentially a multiple of the identity matrix
(since $\lambda_1$ corresponds to the mode
of a uniform shift of all $v_i$,
we can freely set it to $\lambda_2$
without changing the nature of the updating scheme).
%
This shows that in terms of convergence,
the single-bin updating scheme
(as adopted in the WL algorithm)
is the most efficient candidate
of all multiple-bin schemes.



%\section{Numerical results}


%Here


\end{document}
